\documentclass{article}
\usepackage[utf8]{inputenc}

\title{TIME MANAGMENT SYSTEM}
\author{Doddi Abhishek}


\begin{document}

\maketitle






\section{conclusion}
Conventional database object is represented by one row - current state of the object, whereas temporal management
system offers processing object valid data and their changes and progress in time. Developers require access to the
whole information about the evolution of the states during the life-cycle, therefore new paradigm has been created –
temporal system processing. Effective managing temporal data can be very useful for decision making, analyses, process
optimization and can be used in any area – industry, communication systems, medicine, and transport systemsâĂę
However, temporal data management on object level used today does not cover the complexity of the data management
in time. They do not offer sufficient power to manage large volumes of data. A significant aspect is just processing time
and also size.
Our developed system is based on the column attribute level, the whole state is created by the grouping the properties
and states of the attributes. The main advantage is the possibility of data processing with the different granularity.
The developed system is compared with the existing solutions and defines good performance rate – size and time to get
relevant data.
This paper deals with the principles, characteristics and describes implementation methods to provide the complex
temporal data management.
Fully temporal system requires also transaction management, which will be developed and compared in the close future.
The temporal data are usually large; the processing requires sophisticated access methods. In the future development,
we will focus on the various index structures creation, which can improve the performance of the model,



\end{document}
