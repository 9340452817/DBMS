\documentclass{article}
\usepackage[utf8]{inputenc}

\title{TIME MANAGMENT SYSTEM}
\author{Doddi Abhishek}


\begin{document}

\maketitle



\section{ease of use}
The need to manage and model temporal data has come with the first database systems. Developers realized that it
is necessary to store the historical data to provide possibility to restore the database if the database integrity is not correct, the medium is damaged and so on. Backups and log files were considered the main elements of historical data
in the past. However, most programmers used to ignore them, because when using them – it indicated severe problem
and the need to restore last correct version of the database.
   
  Access to historical data using those methods - the image of
the object at a particular time - was complicated, lost too much time, required the administrator intervention and values
were necessary to be reconstructed. Moreover, if the backup was created, the earlier log files were usually deleted.
Thus, the data could be provided only in the time when the database backup was created

\end{document}
