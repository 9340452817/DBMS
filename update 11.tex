\documentclass{article}
\usepackage[utf8]{inputenc}

\title{TIME MANAGMENT SYSTEM}
\author{Doddi Abhishek}


\begin{document}

\maketitle






\section{using statement for data updating}
Updating existing data requires saving old data – not the whole row, but only changed temporal attribute values. The
original table consists of the actual data, so the data manipulation – actual snapshot is easy to get. Historical data
– the snapshot of the whole database, database table or only object – must be also accessible, but are obtained by
passing historical conditions defined by insert, delete or update statement. Thus, the update trigger is started before
update. First of all, the data that are going to be changed, are stored in the table consisting only of the ID of the record
and the value itself.

\section{using statement for data deleting}
The task of the trigger starting before delete is to save old data to the table for deleted objects. The information about
delete is also inserted to the temporal table; ID tab now has the negative value to distinguish whether the data are in
the main or historical tables. Tables for deleted objects have the same structure as corresponded main tables, therefore
they are not modelled in diagrams separately



\end{document}
