\documentclass{article}
\usepackage[utf8]{inputenc}

\title{TIME MANAGMENT SYSTEM}
\author{Doddi Abhishek}


\begin{document}

\maketitle



\section{Classification of Temporal database }

Current issues in this area have not use any classification for temporal model of databases. From this point of view we
suggested classification rule in this form:
α/β/γ ,
where
\item α – represents kind of DBS:
\item• N – No database system support (e.g. file system only)
\item• R – Relational DBS (RDBS)
\item• X – Object relational DBS
\item• O – Object oriented DBS
\item• U – Unspecified DBS
\itemβ – represents kind of the temporal model:
\item• N – Non-temporal (Conventional)
\item• U – Uni-temporal
\item• B – Bi-temporal
\item• M – Multi-temporal
\item• E – Difference defined value of the attribute (ε - epsilon)
\item γ – is represented kind of transaction processing (Online transaction processing - OLTP):
\item• N – Nontransactional
\item• L – OLTP only with logs
\item• O – OLTP with temporal objects
\item• A – OLTP with temporal attributes of the object
For example, temporal database with type R /U/O represents temporal database using RDBS with the uni-temporal
model and with an OLTP
\end{document}
