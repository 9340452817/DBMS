\documentclass{article}
\usepackage[utf8]{inputenc}

\title{TIME MANAGMENT SYSTEM}
\author{Doddi Abhishek}


\begin{document}

\maketitle

\section{OBJECTIVE}

The objective is to develop a time managment system.

\section{ABSTRACT}
Temporal database is an extension of the concept of standard databases which process only current valid data.
Temporal structure is not based only on managing historical data, but it should also model the data, the validity
of which will be in the future in special structures. This paper deals with the temporal structure on object level
in comparison with the column level temporal data. It describes the principles, required methods, procedures,
functions and triggers to provide the functionality of this system. It also defines the possible implementations and
offers the solution to get the snapshot of the database or the object whenever during the existence.

\section{INTRODUCTION}
Massive development of data processing requires access to extensive data using procedures and functions to provide
easy and fast manipulation. The basis is the database technology.
Database systems are the root of any information system and are the most important parts of the information technology.
They can be found in standard applications, but also in critical applications such as information systems for energetics,industry, transport or medicine

Most of the data in the databases represent actual states. However, properties of objects and states are changed over
time - customer changes its status, address; products are modified and updated. If the object state is going to be
changed, data in the database are updated and the database will still contain only currently applicable object states.
But everything has time evolution, thus, history and future that can be useful to store. History management is very
important in systems processing very important or sensitive data; incorrect change would cause a great harm or in the
systems requiring the possibility of restoring the previous states of the database. Therefore, it is necessary to store
not only the current state, but also the previous states and progress. It can also help us to optimize processes or make
future decisions



\end{document}
