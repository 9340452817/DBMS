\documentclass{article}
\usepackage[utf8]{inputenc}

\title{TIME MANAGMENT SYSTEM}
\author{Doddi Abhishek}


\begin{document}

\maketitle





\section{STRUCTURE OF TEMPORAL CLASS}
Temporal extension of the standard conventional (non-temporal) model can be in principle created in two ways - by
defining validity and transaction time. Temporal systems can thus be divided into two basic types:
Uni-temporal system is based on the time of the validity. The instance of the object in the database thus defines the
actual object and also the time which may be defined by an interval (start and end time of the interval), or only by the
beginning. In that case each new instance (record) determines the end of the previous occurrence instance of object
based on the same object. This way - just the begin time of the period - is used in our developed column-level solution.
In this case, however, developer should not forget to manage and report states, during which the object is undefined.
Next table shows the example of the data using the uni-temporal model [5, 10, 11]. The types for interval modelling are
described in [12].
\item Table 1. Uni-temporal table (closed-open representation).
\item  ID BD1 ED1 data
\item2 Jan 2012 Mar 2012 111
\item3 Feb 2012 Nov 2012 123
\item4 Feb 2012 Feb 2013 764
\item2 Mar 2012 Dec 2012 222
\item3 Nov 2012 Feb 2013 890
\end{document}
